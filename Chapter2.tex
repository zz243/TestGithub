\documentclass[12pt]{article}
\linespread{2.0}
\setlength{\parskip}{0.5\baselineskip}
\usepackage{geometry}
\geometry{verbose,letterpaper}
\usepackage{movie15}
\usepackage{hyperref}
\usepackage{graphicx}
\usepackage{epsfig}
\usepackage{longtable}
\usepackage{booktabs}
\usepackage{rotating}
\usepackage{times}
\usepackage{amssymb}
\usepackage{amsmath,bm}
\usepackage{subfigure}
\usepackage[colon]{natbib}
\usepackage{epsfig}
\usepackage{graphicx}
\usepackage{fullpage}
\usepackage{setspace}
\usepackage{authblk}

\usepackage{threeparttable}

\usepackage{multirow}
\usepackage{natbib}

\newcommand{\argmax}{\operatornamewithlimits{argmax}}

\begin{document}
\title{Chapter 2}

\author{Zuoxiang Zhao}

\date{}
\maketitle

\section{introduction}
SFA Meeusen and van den Broeck (1977) and Aigner et al. (1977)


From the statistical point of view, this idea has been implemented by specifying
a regression model characterized by a composite error term in which the classical idiosyncratic disturbance, aiming at capturing measurement error and any other classical
noise, is included together with a one-sided disturbance which represents inefficiency.\footnote{The literature distinguishes between production and cost frontiers. The former represent the maxi-
mum amount of output that can be obtained from a given level of inputs, while the latter characterizes
the minimum expenditure required to produce a bundle of outputs given the prices of the inputs used
in its production.}

where yi represents the logarithm of the output (or cost) of the i-th productive unit,
xi is a vector of inputs (input prices and quantities in the case of a cost frontier) and
$\beta$ is the vector of technology parameters. The composed error term $\varepsilon$ is the sum (or
the difference) of a normally distributed disturbance, vi, representing measurement and
specification error, and a one-side disturbance, ui, representing inefficiency. Moreover,
ui and vi are assumed to be independent of each other and i.i.d. across observations.

The last assumption about the distribution F of the inefficiency term is needed to make
the model estimable. \citet{aigner1977formulation} assumed a Half-Normal distribution,
 while \citet{meeusen1977efficiency} opted for an Exponential
one. Other commonly adopted distributions are the Truncated Normal \citep{stevenson1980likelihood}
(Stevenson 1980) and the Gamma distributions \citep{greene1980maximum} (Greene 1980a,b, 2003).

\citet{pitt1981measurement} were the first to extend model (1-4) to longitudinal data. They
proposed the ML estimation of the following Normal-Half Normal SF model

The generalization of this model to the Normal-Truncated Normal case has been pro-
posed by \citet{battese1988prediction}. As pointed out by \citet{schmidt1984production}, the estimation of a SF model with time invariant inefficiency can also be performed by
adapting conventional fixed-effects estimation techniques, thereby allowing inefficiency
to be correlated with the frontier regressors and avoiding distributional assumptions
about ui. However, the time invariant nature of the inefficiency term has been ques-
tioned, especially in presence of empirical applications based on long panel data sets.
To relax this restriction, \citet{cornwell1990production} have approached the problem proposing
the following SF model with individual-specific slope parameters







\section{Stochastic Frontier Model}

We have estimated three models with time-varying inefficiency: the Normal-Half
Normal Kumbhakar (1990) model (kumb90), a random effect model by means of the
Feasible Generalized Least Squares (FGLS) method, the Cornwell et al. (1990) model
(css90) estimated through the modified-LSDV technique and, finally, the Lee and
Schmidt (1993) model (ls93) estimated using ILS. It is worth noting that the latter
two models are estimated using approaches that do not allow intercept (beta 0) and time
dummies (dyeart) to be simultaneously included into the frontier equation. Finally,
we also considered two models with time-invariant inefficiency, i.e. the uit term boils
down to be uiin equation (36): the first proposed by Schmidt and Sickles (1984) and
estimated without any distributional assumption through the LSDV approach (ss84)
and the second proposed by Pitt and Lee (1981) estimated through ML assuming a Half
Normal inefficiency (pl81).


For more detail on normal-half normal, normal-exponential, and normal-truncated models, see Kumbhakar and Knox Lovell (2000) and Coelli, Prasada Rao, and Battese (1998).

\section{Data}

Because China has no official figures on carbon dioxide emissions released. Therefore, we have to calculate the province-level CO2 emission generated by fossil energy consumption and various industrial production processes, according to the default carbon dioxide emission factors for combustion proposed in 2006 \citeauthor{change20062006} (the UN's Intergovernmental Panel on Climate Change) Guidelines for National Greenhouse Gas Inventories.

目前已被普遍采用的测算资本存量的方法是戈登史密斯(Goldsmith)在1951年开创的永续盘存法。由于中国没有过大规模的资产普查,所以我们在本文中所采用的是在估计一个基准年后运用永续盘存法按不变价格计算各省区市的资本存量。采用相对效率几何递减模型,资本存量的估算可以写作:

或固定资本形成总额(gross fixed capital formation)
我们计算得到了各省固定资本形成总额的经济折旧率 是9.6%




\bibliographystyle{dcu}
\bibliography{ref2}

\end{document}
